So far, we have covered a broad spectrum of topics. Many of these topics are not typical for a calculus II student, while some were.  Instead of taking the typical path through calculus II, then calculus III, we are venturing in a way that allows us to tell a story and build a firm foundation in mathematics specifically used throughout chemistry (and especially physical chemistry).  Typically, students in the calculus sequence learn differentiation and integration in a first course.  In a second course, students learn about sequences, series, power series, and Taylor series.  The third course entails learning calculus in higher dimensions. Specifically, one concentrates on calculus in 3-dimensional space since this is (arguably) the most physically meaningful to us.  Following the calculus sequence comes a course in differential equations where students are granted a handbook of techniques for solving different types of equations.

We approached this in a completely different way.  We began by studying complex numbers as this field of numbers allows us to factor polynomials.  The complex numbers also formed a vector space, and we briefly investigated this structure.  However, the main goal was to use these complex numbers in order to be able to solve many different first and second order differential equations.  These differential equations arose by studying systems that change over time. For example, we saw the harmonic oscillator equation arise from a spring/mass system. We also saw that chemical reactions were nicely described by differential equations. Those equations were describing systems that evolved over time and were given with initial function values at time zero. Another way differential equations arose was via boundary value problems. In particular, we studied the free particle in a one-dimensional box and came across many new concepts that resurface later.

Quickly, one sees that the techniques we used to solve differential equations were not all powerful. Thus, we sought out a new technique to solve more equations.  Eventually, we arrived at power series which provided us with newfound abilities to compute.  For example, one needs a tool such as a power series to compute values to the function $\sin(x)$!  Power series then proved as indispensable tools for solving more differential equations than we were previously able to work with.  When we still had trouble with a specific equation, we could then use a Taylor series to find polynomial approximations of terms in a differential equation and from there we could solve this equation using power series.  This part of the class was closer to a typical calculus II course but came with the added bonus of solving differential equations as well.  Even in a first course on ordinary differential equations, one may not solve equations using a power series! 

Finally, the last portion of this class came as a preparation to deal with higher dimensional spaces.  Understanding vector spaces and how they transform is a necessary building block to studying calculus in higher dimensions.  However, linear algebra can be done in more generality than we have covered here.  In this generality, we can see how topics we have previously worked with fit in as well.  For example, we spoke of linear differential equations and one can show that the set of solutions to a linear differential equation forms a vector space! Either way, the lessons that linear algebra teaches us about geometry and using geometrical tools to solve problems is highly important.  This can lead one to considering abstracting algebra further and seeing if that can help solve problems as well.  

Next, we move onto the sequel of Math 271 which is Math 272 where we will spend time learning calculus in higher dimensions.  In order to model the physical world, this is completely necessary.   The topics covered in 271 will lead straight into the new mathematics awaiting us in 272.  Keep these previous chapters as a reference as they are now prerequisite material!